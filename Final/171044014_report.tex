\documentclass[20pt]{article}
\usepackage{graphicx}
\usepackage[inner=2.0cm,outer=2.0cm,top=2.5cm,bottom=2.5cm]{geometry}
\usepackage{setspace}
\begin{document}

\title{%
  CSE 344 - System Programming \\
  \large Final}

\author{Harun ALBAYRAK - 171044014}

\maketitle

\Large
\section{How did I solve this problem?}

Server:\\\\
Firstly, i created a daemon process with fork().\\
In the daemon process, first, i check the errors. If the number of arguments is not equal to 9, i print an error and exit.\\
After that, i handle the command line arguments. First, i check whether it is incompatible or not every flag. 
\\\\
After that, i read the csv file and store its data in the Linked List. Its columns and rows are stored in separate linked lists. I implemented custom methods for optimization in linked list. In this way, the running time of the program is reduced.
\\\\
After reading the csv file I open a socket to accept the data. The socket uses TCP(Transmission Control Protocol).
\\\\
After opening the socket I create as many threads as the number of pool size in the command line. 
\\\\
The threads waits a connection and a query. After coming a query, firstly, the query is parsed. And if the query type is SELECT, server\_reader() function is called. If the query type is UPDATE server\_writer() function is called. There is a reader-writer paradigm in here. 
\\\\
I use condition variables and a mutex in order to solve reader-writer problem. 
\\\\
I keep time before server\_reader() or server\_writer() function and after server\_reader() or server\_writer() function. In this way, i find the time between them. This gives the runtime of query.  
\\\\
If the query type is SELECT i create the message which is sent to client in a separate function. If the query type is UPDATE i create the message in server\_writer() function.
\\\\
After creating the message, the message is sent to relevant client. And the thread waits a new query after it sleep 0.5 seconds.
\\\\\\
Client: \\\\
In the client, first, i check the errors in the same way. If the number of arguments is not equal to 9, i print an error and exit. After that i check the flags and its arguments.
\\\\
After checking the errors and stored its arguments in a struct, i open a socket and i connect to a socket. And the query is sent to the socket.
\\\\
The server gets the query and creates a message. The message is sent to the client. The client prints it to the console.
\\\\
This is repeated until the queries are finished.\\\\

\section{My Design Decisions}
I create a 'ServerArguments' struct so that i store server's command line arguments.
\\\\
I create a 'ClientArguments' struct so that i store clients' command line arguments.
\\\\
I create a 'Query' struct so that i store the query's data.
\\\\
I create a 'Message' struct so that server sends to client.
\\\\
I create a 'QueryType' enum for 'Query' struct.
\\\\
I use some global variables to solve some issues.
\\\\
I use linked lists to store data in the csv file.\\
Why?\\
Because the data is appended to the linked list only constant( O(1) ) time.
\\\\

\section{Requirements I achieved and which I have failed}
I think I achieved almost all the requirements. However, I may not have been able to achieve some requirements.
\\
\section{My Files}
serverMain.c $\Rightarrow$ The Server Main C File \\
serverFunctions.c $\Rightarrow$ The Server Functions \\
serverFunctions.h $\Rightarrow$ The Server Functions' Headers \\
clientMain.c $\Rightarrow$ The Client Main C File \\
clientFunctions.c $\Rightarrow$ The Client Functions \\
clientFunctions.h $\Rightarrow$ The Client Functions' Headers \\
linkedList.c $\Rightarrow$ Linked List Functions \\
linkedList.h $\Rightarrow$ Linked List Functions' Headers \\
helper.c $\Rightarrow$ Helper Functions \\
helper.h $\Rightarrow$ Helper Functions' Headers \\
171044014\_report.pdf $\Rightarrow$ The Report PDF \\
171044014\_report.tex $\Rightarrow$ The Report Latex file \\
Makefile $\Rightarrow$ The Makefile \\

\section{Some screenshots from the program}
\begin{figure}[h!]
  \includegraphics[width=\linewidth]{final_2.png}
  \caption{The server}
  \label{fig:code}
\end{figure}

\begin{figure}[h!]
  \includegraphics[width=\linewidth]{final_3.png}
  \caption{Client 2}
  \label{fig:code}
\end{figure}

\begin{figure}[h!]
  \includegraphics[width=\linewidth]{final_4.png}
  \caption{Client 1}
  \label{fig:code}
\end{figure}

\begin{figure}[h!]
  \includegraphics[width=\linewidth]{final_5.png}
  \caption{Client 2}
  \label{fig:code}
\end{figure}

\begin{figure}[h!]
  \includegraphics[width=\linewidth]{final_6.png}
  \caption{Client 1}
  \label{fig:code}
\end{figure}

\end{document}

