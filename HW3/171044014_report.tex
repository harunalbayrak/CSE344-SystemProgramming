\documentclass[20pt]{article}
\usepackage{graphicx}
\begin{document}

\title{%
  CSE 344 - System Programming \\
  \large Homework \#3}

\author{Harun ALBAYRAK - 171044014}

\maketitle

\Large
\section{How did I solve this problem?}
Firstly, i checked number of arguments. If an argument flag is used more than one or not used at all, i printed an error and exit. \\
If –b flag consist of negative number, I printed an error and exit. 
\\\\
After check these situations, i handle it the arguments in handleTheArgs() function.  It is also my important function. Firstly i open "fileWithFifoNames" file and i read line by line. While reading, I also put it into an array. 
\\\\
After that, i put the processes into shared memory with their pid, number of switches, total switch, and fifo path which randomly chosen. 
\\\\
I also increment by 1 at every creating process the named semaphore in order to check enough process to run program.
\\\\
After creating enough process to run program(Enough processes' number is number of line in the given file.), I check that whether the all processes' hands is empty. If so, i print an error and exit.
\\\\
After these phases, if the potato number(in -b flag) is zero, i created a FIFO and waiting for potatoes that the another processes sent. I use named semaphore at the beginning and at the end for synchronization. 
\\\\
If the potato number(in -b flag) is more than one, i send the potatoes to other processes. I use named semaphore at the beginning and at the end for synchronization too.  
\\\\
\section{My Design Decisions}
I create two struct as "Process" and "Request". These structs are used by functions. Also shared memory keep pointer of "Process" struct.
\\\\
If all process are created with empty hands, the program is terminated.
\\\\
Similarly, If all process are created with potatoes, the program is terminated. ​
\\\\
\section{Requirements I achieved and which I have failed}
I think I achieved almost all the requirements. However, I may not have been able to achieve some requirements.

\section{My Files}
171044014\_helper\_hw3.h $\Rightarrow$ The helper functions \\
171044014\_hw3.c $\Rightarrow$ The Main C File \\
171044014\_report.pdf $\Rightarrow$ The Report PDF \\
171044014\_report.tex $\Rightarrow$ The Report Latex file \\
Makefile $\Rightarrow$ The Makefile \\

\section{Some screenshots from the program}
\begin{figure}[h!]
  \includegraphics[width=\linewidth]{hw3_0.png}
  \caption{The file}
  \label{fig:code}
  \includegraphics[width=\linewidth]{hw3_1.png}
  \caption{The process 1}
  \label{fig:code}
  \includegraphics[width=\linewidth]{hw3_2.png}
  \caption{The process 2}
  \label{fig:code}
  \includegraphics[width=\linewidth]{hw3_3.png}
  \caption{The process 3}
  \label{fig:code}
\end{figure}

\begin{figure}[h!]
  \includegraphics[width=\linewidth]{hw3_4.png}
  \caption{The process 1}
  \label{fig:code}
  \includegraphics[width=\linewidth]{hw3_5.png}
  \caption{The process 2}
  \label{fig:code}
  \includegraphics[width=\linewidth]{hw3_9.png}
  \caption{The process 3}
  \label{fig:code}
\end{figure}

\begin{figure}[h!]
  \includegraphics[width=\linewidth]{hw3_6.png}
  \caption{The process 1: Since all processes have potatoes the program is not work.}
  \label{fig:code}
  \includegraphics[width=\linewidth]{hw3_7.png}
  \caption{The process 1: Since all processes have potatoes the program is not work.}
  \label{fig:code}
  \includegraphics[width=\linewidth]{hw3_8.png}
  \caption{The process 1: Since all processes have potatoes the program is not work.}
  \label{fig:code}
\end{figure}

\end{document}
